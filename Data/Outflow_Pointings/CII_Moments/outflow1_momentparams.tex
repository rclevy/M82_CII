\begin{deluxetable*}{cccccccccc}
\tablecaption{Properties of \CII\ spectra in the outflow \label{tab:outflowfits}}
\tablehead{Pixel Number & R.A. & Decl. & ${\rm I_{peak}}$ & ${\rm I_{int}}$ & ${\rm V_0}$ & $\sigma_{\rm V}$ & rms & ${\rm log_{10}M_{C^{+}}}$\\& (J2000 hours) & (J2000 degrees) & (mK) & (K~km~s$^{-1}$) & (km~s$^{-1}$) & (km~s$^{-1}$) & (mK) & (${\rm log_{10}M_\odot}$)}\startdata
0 & $9^\mathrm{h}55^\mathrm{m}54.6^\mathrm{s}$ & $+69^\circ39{}^\prime24.3{}^{\prime\prime}$ & 72.0 $\pm$ 14.8 & 10.8 $\pm$ 0.1 & 143.3 $\pm$ 10.2 & 94.2 $\pm$ 16.9 & 8.3 & 3.7 $\pm 0.2$\\
1 & $9^\mathrm{h}55^\mathrm{m}56.2^\mathrm{s}$ & $+69^\circ39{}^\prime46.9{}^{\prime\prime}$ & 98.6 $\pm$ 33.4 & 21.2 $\pm$ 0.3 & 141.2 $\pm$ 10.3 & 96.0 $\pm$ 19.7 & 9.4 & 4.0 $\pm 0.1$\\
2 & $9^\mathrm{h}55^\mathrm{m}56.9^\mathrm{s}$ & $+69^\circ39{}^\prime19.8{}^{\prime\prime}$ & 88.9 $\pm$ 28.2 & 13.7 $\pm$ 0.3 & 144.9 $\pm$ 10.4 & 89.3 $\pm$ 17.3 & 9.6 & 3.8 $\pm 0.2$\\
3 & $9^\mathrm{h}55^\mathrm{m}55.3^\mathrm{s}$ & $+69^\circ38{}^\prime59.1{}^{\prime\prime}$ & 77.5 $\pm$ 31.0 & 16.4 $\pm$ 0.3 & 165.1 $\pm$ 10.5 & 111.8 $\pm$ 93.7 & 6.9 & 3.9 $\pm 0.1$\\
4 & $9^\mathrm{h}55^\mathrm{m}53.4^\mathrm{s}$ & $+69^\circ39{}^\prime00.0{}^{\prime\prime}$ & --- & --- & --- & --- & 6.8 & ---\\
5 & $9^\mathrm{h}55^\mathrm{m}52.5^\mathrm{s}$ & $+69^\circ39{}^\prime30.1{}^{\prime\prime}$ & --- & --- & --- & --- & 16.8 & ---\\
6 & $9^\mathrm{h}55^\mathrm{m}54.1^\mathrm{s}$ & $+69^\circ39{}^\prime52.3{}^{\prime\prime}$ & 203.0 $\pm$ 27.4 & 23.3 $\pm$ 0.3 & 180.2 $\pm$ 10.2 & 77.4 $\pm$ 247.0 & 11.5 & 4.1 $\pm 0.1$\\
\enddata
\tablecomments{The R.A. and Decl. of the center of each pixel are given. The other columns show the moments of the spectra including the peak intensity (${\rm I_{peak}}$), the integrated intensity (moment 0; ${\rm I_{int}}$), the mean velocity (moment 1; ${\rm V_0}$), the linewidth (moment 2, $\sigma_{\rm V}$), and corresponding uncertainties. The moments are calculated over a fixed velocity range spanning -$100-400$~\kms. rms is the root-mean-square noise of the spectrum calculated outside of the velocity range used for the moments. If the moments cannot be calculated because the line is not detected, then the rms is calculated over the entire bandpass. All temperatures refer to T$_{\rm mb}$. ${\rm M_{C^{+}}}$ is the mass in C$^+$ inferred from the spectrum; see Appendix \ref{app:density_calc} for details of this calculation.} 

\end{deluxetable*}
